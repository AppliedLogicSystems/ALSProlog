\oddsection{Application Packaging}
\label{Packaging}

For delivery of applications programs to users, it is desirable to be
able to package all elements of the program --- ALS Prolog run-time system,
foreign predicates, and Prolog predicates --- together in one executable file.
ALS Prolog provides tools for achieving this goal.  Currently the tools are
only available for systems utilizing the Intel 80386 chip (UNIX or DOS).  However,
the tools will be extended to other computer architectures in the future.

Application packaging in the ALS Prolog environment involves 
creating object files for the Prolog programs and linking them with 
any required foreign object files and the
ALS Prolog library to get an executable image.  The packaging can be done incrementally,
with subsequent packages being built on top of previous packages.

There are two principal steps to create an application package using
the ALS Prolog packaging system 
These steps are as follows:

\begin{enumerate}
\item Create object files representing the Prolog programs.
\item Link these object files together with any required foreign object files and with the 
ALS Prolog library to create a new executable image.
\end{enumerate}

\subsection{How to Create Package Files}

\subsubsection{Creating A Package}

To create package files, first you have to run the extended ALS Prolog 
with packaging predicates. 
This extended ALS Prolog image contains all predicates in ALS Prolog and
predicates used to create package files.
In fact, this extended image is created by linking ALS prolog library
with objects files representing Prolog builtins and predicates in
the packaging system.


If your package will include foreign predicates, presumably you have previously linked 
them with the ALS Prolog library to create a customized version of ALS Prolog;  
let us assume it is called YourProlog.  Then, instead of running the standard 
ALS Prolog, run YourProlog and consult the appropriate packaging file.

After these packaging files are consulted, you are ready to use the
ALS packaging system.   Next, consult the Prolog files you wish to package.  You are now
ready to start generating the required object files.  This is done by invoking one of the
top-level packaging predicates described in the next section.

\subsubsection{Packaging Predicates}

The principal top-level predicate in the packaging system is \verb|package/12|.
This predicate is called to create package files:

\begin{verbatim}
    package(Procs,PckgName,PckgFile,PckgTokTblFile,StartPred,PrePckgName,
            DefProcSwch,TokTblSwch,RawDataSize,RelTblSize,SymTblSize,StrTblSize)
\end{verbatim}

The roles of the arguments to \verb|package/12| are as follows:

\begin{description}
\item[Procs] \ \\ 
This argument represents procedures which will be packaged. 
This can be either a term in one of the following forms
\begin{verbatim}
    M:P/A	
    P/A
    P
\end{verbatim}
or can be a list of any terms of these forms.
Moreover, any part of terms above can be an uninstantiated variable.
For example, the following expression
\begin{verbatim}
    [user:_,m1:p/1]
\end{verbatim}
represents all procedures in the module \verb|user| together with the procedure \verb|p/1|
from the module \verb|m1|. 
In this setting,  \emph|procedure| is not restricted to predicates
defined in Prolog, but refers to any name table entry in the system,
including foreign predicates.
In other words, any name table entry can be used as a procedure in the packaging system. 

\begin{quotation}
\emph{IMPORTANT NOTE:}\\
The set of procedures defined by this argument should constitute
a \emph{closed} set of procedures in the sense that every procedure
called by one of the procedures described in the \verb|Procs| argument
must either occur in the list, or its definition must be contained in
the underlying package on which you are building, or it must a primitive
builtin defined by the ALS Prolog library.
This means that if you are 
packaging a procedure \verb|p| and the following is one of its clauses,
\begin{verbatim}
    p(X,Y) :- q(X), r(Y).
\end{verbatim}
you must package procedures \verb|q| and \verb|r| in the same package,
or they must be primitive procedures in the ALS Prolog library, or they
must be defined in the package on which you are building.

It is not necessary, but it will be good practice to package
each module separately since procedures defined by
the term \verb|modulename:_| are guaranteed to be a closed set
of procedures as long as any modules which \verb|modulename| uses
have been previously packaged or are packaged along with
\verb|modulename|.
\end{quotation}
\item[PckgName] \ \\
This argument is the name of the package to be created.
Later, the created package can be referred by this name.
The name can be any Prolog symbol, but it is advisable that it
begin with a letter and contain only
alphanumeric characters and the underbar character \_. 
\item[PckgFile] \ \\
The name of the package object code file which will store 
the object definitions of the procedures listed in \verb|Procs|
(their name table entries and their clauses if they are 
Prolog predicates).
\item[PckgTokTblFile] \ \\
The name of the object code file which will store the object definitions of
the token table, the module table, the default use module table, and the default
procedure table.
\item[StartPred] \ \\
This the name of a zero-arity predicate name which will be executed
after the package is loaded. Normally, after a package is loaded,
the system enters the usual Prolog top-level environment. 
\verb|StartPred|
provides an option to the developer to decide whether he/she wants to 
be in Prolog top-level environment or to execute the indicated zero-arity predicate.
Of course, in the second case the system executes the zero-arity
prdicate before it enters the usual Prolog top-level
environment.  Normally, if any application is being built out of a sequence of
packages, all but the last use the default, while the last sets this argument
to a zero-arity predicate which starts the application being packaged.
The values of this argument are -1 (the default case), \verb|P| or
\verb|M:P| where \verb|M| is the module name and \verb|P| is the name of a 
zero-arity predicate in that module.
\item[PrePckgName] \ \\
This is the name of the previous package which is the father
of the package which is going to be created. An complete application
package consists of a list of package files which contain
procedures and a package token table file (the token table file generated by
the last package in the sequence).
The package files are related to each other
with father relation. The default value of \verb|PrePckgName| is 
the currently loaded package in the system. If there is no
previous package (i.e., one is starting from ALS Prolog with the first package in the sequence), 
the value of this argument is -1. Normally
all user packages will have a previous package. The user should
guarantee that the father of the package which is going to be created
has been already loaded in the system. 
\item[DefProcSwch] \ \\
In ALS Prolog, there are default procedures which should
explicitly imported in each module when a module is loaded.
(These include predicates exported from builtins, and module closures
defined in other modules which are used by the current module.)
The name table entries of these procedures should be saved in
the current package if they are not already saved in one of its ancestor
packages. If this switch is 1, all name table entries
of these default procedures in all modules are saved in the package.
If the switch is -1 (default case), they are not saved. 
The user normally will set this switch to 1 when he/she packages a new module
or predicates whose module closures defined by him/her.
\item[TokTblSwch] \ \\
This is the switch which determines whether the package token table
file will be created or not. If it is 1 (default case), the file
is created. If it is -1, it won't be created. In each application
package, the package token table of the last package is used
as the package token table of that application package.
If we are creating a package file and we don't have any intention to
use it as a final application package, we don't need to create its 
package token table file. Our intention here is to use the package as 
a father of another package which is going to be created in the same
session (without leaving the extended ALS Prolog image).
\item[RawDataSize] \ \\
This switch determines how much memory will be allocated for raw data area
when package .o files are created. When it is 0 (default case), 
a predetermined size (256K Bytes) is used by the system.
If the user sets this option to a positive number, \emph{w}, the system
allocates \emph{w} KBytes memory for raw data area.  
The maximum value for this option is 8192 (8 MBytes).
\item[RelTblSize] \ \\
This switch determines how many entries will be allocated for relocation entries
when package .o files are created. When it is 0 (default case), 
a predetermined number of relocation entries (8192 entries) 
are allocated by the system.
If the user sets this option to a positive number, \emph{w}, the system
allocates \emph{w}x1024 entries for the relocation table.  
The maximum value for this option is 256 (256x1024 entries).
\item[SymTblSize] \ \\
This switch determines how many entries will be allocated 
for symbol table entries
when package .o files are created. When it is 0 (default case), 
a predetermined number of symbol table entries (8192 entries) 
are allocated by the system.
If the user sets this option to a positive number, \emph{w}, the system
allocates \emph{w}x1024 entries for the symbol table.  
The maximum value for this option is 256 (256x1024 entries).
\item[StrTblSize] \ \\
This switch determines how much memory will be allocated 
for string table 
when package .o files are created. When it is 0 (default case), 
a predetermined size (128K Bytes) is used by the system.
If the user sets this option to a positive number, \emph{w}, the system
allocates \emph{w} KBytes memory for string table.  
The maximum value for this option is 4096 (4 MBytes).
\end{description}

Other versions of the package predicates with fewer arguments are
available. In these cases, default values for the missing arguments
are supplied. These predicates are as follows.

\begin{verbatim}
     package(Procs,PckgName)
     package(Procs,PckgName,PckgFile,PckgTokTblFile)
     package(Procs,PckgName,PckgFile,PckgTokTblFile,StartPred,PrePckgName,
             DefProcSwch,TokTblSwch)
\end{verbatim}
	
Default values for missing arguments are as follows.

\begin{itemize}
\item
PckgFile : \verb|<package name>.s|
\item
PckgTokTblFile : \verb|<package name>tok.s|
\item
StartPred : -1 
\item
DefProcSwch : -1 ( default procedure names won't be included)
\item
TokTblSwch : 1 ( the token table file will be created )
\item
PrePckgName : name of the currently loaded package in the system.
\item
RawDataSize : 0 ( 256 KBytes)
\item
RelTblSize : 0 ( 8192 entries)
\item
SymTblSize : 0 ( 8192 entries)
\item
StrTblSize : 0 ( 128 KBytes )
\end{itemize}


\subsection{Example: Creating a Package out of the Packaging Files}

To create a package consisting of the packaging file itself, we must first 
run ALS Prolog, and consult the packaging file, \verb|pckg.obp|.

After this  file is consulted, we are ready to use the
ALS Packaging system. At this point, there is no currently loaded package in the system
(we have simply loaded the code defining the packaging system), and
there are three modules present in the system:

\begin{enumerate}
\item
user --- contains only default procedures which are included by every module in the system.
\item
builtins --- contains all builtins predicates
\item
packaging --- contains all packaging predicates
\end{enumerate}

We can create a single package which contains all predicates defined in
modules above, or we can incrementally create three separate package files. 
We will create three separate packages in sequence: \verb|tuser|, \verb|tblt|, and \verb|tpckg|. 
These packages will be organized in the father relation as follows, with the father package
appearing above its descendent:

\begin{verbatim}
    tuser
      ^
      |
      |
    tblt
      ^
      |
      |
    tpckg
\end{verbatim}

Creation of these packages will be incremental.
First, the package \verb|tuser| is created to package predicates in the
module user. Its parent will be an empty package. Second,
we create the package \verb|tblt| which will contain procedures defined
in the module builtins. Finally, the package \verb|tblt| will be created to package
the packaging procedures.

To create the first package \verb|tuser|, the following goal can be executed.

\begin{verbatim}
    package(user:_,tuser,-1,-1,-1)
\end{verbatim}

The file \verb|tuser.s| will be created to contain all procedures in 
the module builtins. Since we are not planning to use this package file
as an application package, we don't need to create a package token
table file. Even though the default procedures from all modules won't be saved,
the default procedures in the module \verb|user| will be saved since we are
saving all procedures in that module.

Now, without leaving the Prolog environment, we create the second package
\verb|tblt| as the descendent of the package \verb|tuser|. We can create this package 
as follows.

\begin{verbatim}
    package(builtins:_,tblt,-1,-1,1,tuser)
\end{verbatim}

The files \verb|tblt.s| and \verb|tblttok.s| will be created to contain all procedures
in the module \verb|builtins| together with the token table and other information about
the package, respectively. The reason that we create a token table file in this case is 
that we are planning to use the packages \verb|tuser| and \verb|tblt| together as an application
package. The combination of these files \verb|tuser.s|, \verb|tblt.s|, \verb|tblttok.s|
and the ALS Prolog library represent a packaged version of the ALS Prolog system.

Now we can create our third package file on the top of the second one:

\begin{verbatim}
    package(packaging:_,tpckg,-1,-1,1,tblt)
\end{verbatim}

The file \verb|tpckg.s| will contain the procedures from the packaging
system, and the file \verb|tpckgtok.s| will contain the token table and
other information about the package. The combination of files \verb|tuser.s|,
\verb|tblt.s|, \verb|tpckg.s|, \verb|tpckgtok.s| and the ALS Prolog library will represent
a packaged ALS Prolog together with the packaging system.

\subsection{Formats of Package Files}

An application package consists of a list of package files containing object definitions of
procedures, together with an object file definiing the package token table.
Package files which contain procedures are related to each other 
by the father relation. An application package will look like following:

\begin{verbatim}
    package_file_1
         ^
         |
         |
    package_file_2
         ^
         |
         |
         .
         .
         .
         ^
         |
         |
    package_file_n    <--   package_token_table_file
\end{verbatim}

The package token table for an application package (there is only one
package token table for each application package) contains top level
information about the application package. The information in this file
actually defines the application package. The package token table 
defines five gloabal variables used by ALS Prolog packaging system.

\begin{description}
\item[system\_pckg] \ \\
Name of the root package in the chain of packages
\item[pckg\_start\_pred] \ \\
Package start predicate
\item[pckg\_modtbl] \ \\
Package module table\\
Package default use module table\\
Package default procedure table\\
\item[pckg\_toktbl] \ \\
Package token table
\item[pckg\_toktbl\_size] \ \\
Number of tokens in package token table
\end{description}

All of these five variables are made public, so that the system can obtain their
values when the application package is loaded.
These variables store the information present in the system tables when the package was created.
When the application package is loaded, the system tables are restored
from the tables in this file.

A package file of an application package contains packaged name table
entries and the clauses of the procedures. It defines only one global variable
which holds information about that package. The format of a package
file will be as follows.

\begin{verbatim}
pckg_code_start:

	<packaged name table entries of procedures>

	<a dummy used block>
	<packaged clauses of procedures>
	<a dummy used block>

pckg_code_end:

pckg_ntbl:
	<number of name table entries in package>
	<list package name table entries>

<package name>:
	<previous package>
	<location of pckg_code_start>
	<location of pckg_code_end>
	<package name>
\end{verbatim}

The variable \verb|<package name>| is the only public variable in this file.
All other variables are local to this file. The vVariables \verb|pckg_code_start|
and \verb|pckg_code_end| mark the beginning and end of the code area of the package.
This information is used when another package is created on the top
of this package. The dummy used blocks mark the beginning and end of packaged
clauses, and they can be used when packaged clauses are retracted,
so the space occupied by a retracted packaged clause can be reclaimed.

The variable \verb|pckg_ntbl| contains a list of pointers to packaged name
table entries. When an application package is loaded, the system name 
table is initialized from the information in the package name table.

The variable \verb|<previous package>| is a pointer to the previous package, so that
the previous package can be accessed from this package.
The variables \verb|<location of pckg_code_start>|, \verb|<location of pckg_code_end>|, and
\verb|<package name>| are used when another package is created from this 
package. 

\subsection{Packaged Name Table Entries}

A packaged name table entry is an object procedure which exactly represents 
the corresponding name table entry. 
The size of the object code will be exactly size of a name table entry. 
In fact,
the idea is to use this object procedure as a name table entry.
A packaged name table entry will look like the following (in MS-DOS):

\begin{verbatim}
		align 	4
		dd	<location of name table entry>
<procedure name> proc near
		<name table header>
		<overflow code>
<procedure name>_call:
		<call entry code>
<procedure name>_exec:
		<exec entry code>
<procedure name>_code:
		<code entry code>
<procedure name> endp
\end{verbatim}

The locations \verb|<procedure name>_call|, \verb|<procedure name>_exec|, and
\verb|<procedure name>_code| are made public so that they can be accessed
by other packages. \verb|<procedure name>| is a symbol which is constructed
from \verb|package name| and \verb|location of name table entry|.
The original location of the name table entry which is used in 
\verb|<procedure name>| is saved immediately before a packaged
name table entry, so that it can be used to produce the same \verb|<procedure name>|.



